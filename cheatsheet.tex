\documentclass[10pt,landscape]{article}
\usepackage{multicol}
\usepackage{calc}
\usepackage{ifthen}
\usepackage{amsmath}
\usepackage{verbatim}
\usepackage[landscape]{geometry}

% To make this come out properly in landscape mode, do one of the following
% 1.
%  pdflatex latexsheet.tex
%
% 2.
%  latex latexsheet.tex
%  dvips -P pdf  -t landscape latexsheet.dvi
%  ps2pdf latexsheet.ps


% If you're reading this, be prepared for confusion.  Making this was
% a learning experience for me, and it shows.  Much of the placement
% was hacked in; if you make it better, let me know...


% 2008-04
% Changed page margin code to use the geometry package. Also added code for
% conditional page margins, depending on paper size. Thanks to Uwe Ziegenhagen
% for the suggestions.

% 2006-08
% Made changes based on suggestions from Gene Cooperman. <gene at ccs.neu.edu>


% To Do:
% \listoffigures \listoftables
% \setcounter{secnumdepth}{0}


% This sets page margins to .5 inch if using letter paper, and to 1cm
% if using A4 paper. (This probably isn't strictly necessary.)
% If using another size paper, use default 1cm margins.
\ifthenelse{\lengthtest { \paperwidth = 11in}}
	{ \geometry{top=.5in,left=.5in,right=.5in,bottom=.5in} }
	{\ifthenelse{ \lengthtest{ \paperwidth = 297mm}}
		{\geometry{top=1cm,left=1cm,right=1cm,bottom=1cm} }
		{\geometry{top=1cm,left=1cm,right=1cm,bottom=1cm} }
	}

% Turn off header and footer
\pagestyle{empty}
 

% Redefine section commands to use less space
\makeatletter
\renewcommand{\section}{\@startsection{section}{1}{0mm}%
                                {-1ex plus -.5ex minus -.2ex}%
                                {0.5ex plus .2ex}%x
                                {\normalfont\large\bfseries}}
\renewcommand{\subsection}{\@startsection{subsection}{2}{0mm}%
                                {-1explus -.5ex minus -.2ex}%
                                {0.5ex plus .2ex}%
                                {\normalfont\normalsize\bfseries}}
\renewcommand{\subsubsection}{\@startsection{subsubsection}{3}{0mm}%
                                {-1ex plus -.5ex minus -.2ex}%
                                {1ex plus .2ex}%
                                {\normalfont\small\bfseries}}
\makeatother

% Define BibTeX command
\def\BibTeX{{\rm B\kern-.05em{\sc i\kern-.025em b}\kern-.08em
    T\kern-.1667em\lower.7ex\hbox{E}\kern-.125emX}}

% Don't print section numbers
\setcounter{secnumdepth}{0}


\setlength{\parindent}{0pt}
\setlength{\parskip}{0pt plus 0.5ex}


% -----------------------------------------------------------------------


\begin{document}

\raggedright
\footnotesize
\begin{multicols}{3}


% multicol parameters
% These lengths are set only within the two main columns
%\setlength{\columnseprule}{0.25pt}
\setlength{\premulticols}{1pt}
\setlength{\postmulticols}{1pt}
\setlength{\multicolsep}{1pt}
\setlength{\columnsep}{2pt}

\begin{center}
     \Large{\textbf{Numeric Cheat Sheet}} \\
\end{center}

\section{Zerlegung}
\subsection{Cholesky}
\begin{tabular}{@{}l}
Die Cholesky-Zerlegung erstellt eine Matrix $G$ und $G^T$,\\
so dass folgendes gilt:
\end{tabular}
$\begin{bmatrix}
 g_{11}     & &  \\
 \vdots & \ddots & \\
 g_{gk1}      & \cdots & g_{kk}
\end{bmatrix}
\begin{bmatrix}
 g_{11} & \cdots    & g_{k1} \\
        & \ddots    & \vdots \\
        &           & g_{kk}
\end{bmatrix} =
\begin{bmatrix}
    a_{11} & \cdots    & a_{1k} \\
    \vdots & \ddots    & \vdots \\
    a_{k1} & \cdots    & a_{kk}
\end{bmatrix}$

\paragraph{Formeln} \ \\
\begin{tabular}{@{}l}
$g_{11} = \sqrt{a_{11}}$ \\
$g_{kk} = \sqrt{a_{kk} - \displaystyle\sum_{j=1}^{k-1} g_{kj}^2}$ \\
$g_{ik} = \frac{a_{ik} - \displaystyle\sum_{j=1}^{k-1} g_{ij}g_{kj} }{g_{kk}}$ \\
\end{tabular} \\

\subsection{LR ohne Pivot}
Die LR-Zerlegung ist im Grunde ein Gauss, bei welchem die Umformungen von einer Matrix $A$ nach $R$ in einer Matrix $L$ gespeichert werden.\\
Anschliessend werden folgende LGS gel\"ost:\\
\begin{tabular}{@{}ll@{}}
$LRx$   & $= b$ \\
$Lc$    & $= b$ \\
$Rx$    & $= c$
\end{tabular}
\paragraph{Beispiel}�\ \\
$\begin{matrix}
I \\
II
\end{matrix}
\begin{bmatrix}
3 & 2\\
6 & 6
\end{bmatrix} \Rightarrow
\begin{matrix}
I \\
II^{'}
\end{matrix} 
\begin{bmatrix}
3 & 2\\
0 & 2
\end{bmatrix} =: R$ \\
$II^{'} = II - \frac{6}{3}I ,q_1 = 6/3 = 2$ \\
$L := \begin{bmatrix}
1 \\
2 & 1
\end{bmatrix}$ 

\begin{comment}
\subsection{LR mit Pivot}
Wird zus\"atzlich eine Pivotwahl vorgenommen, so wird die Wahl in einer Permutationsmatrix zwischengespeichert.\\
Dann wird es irgendwie ziemlich h\"asslich, da sowohl $b$ und $x$ auch permutiert werden m\"ussen.\\
\begin{tabular}{@{}ll@{}}
$LRx$   & $= b$ \\
$PLc$    & $= b$ \\
$Rx$    & $= c$
\end{tabular}
\end{comment}

\section{Interpolationspolynomen}
\subsection{Newton}
In der Newton-Darstellung entsteht ein Polynom dessen Koeffizienten durch divergierten Differenzen definiert sind. \\
Die Polynome sind die Newton-Polynome $w_i(x)$. \\
$p(x) = f[x_0] w_0(x) + f[x_0,x_1] w_1(x) + ... + f[x0,...,x_n] w_n(x)$
\subsubsection{Newton-Polynome}
Die Newton-Polynome haben alle folgende Form:\\
$w_i(x) = \displaystyle\sum_{j=0}^{i-1} (x - x_j)$
\subsubsection{dividierte Differenzen}
\begin{tabular}{@{}ll@{}}
$f[x_i]$ & $= f_i$ \\
$f[x_0,x_1]$ & $=\frac{f[x_0]-f[x_1]}{x_0 - x_n}$ \\
$f[x_0,...,x_n]$ & $= \frac{f[x_0,...,x_{n-1}] - f[x_1,...,x_n]}{x_0 - x_n}$
\end{tabular}
$\begin{matrix}
    f_0 := f[x_0] \\
                & \searrow \\
    f_1 := f[x_1] & \rightarrow   & f[x_0,x_1] \\
                & \searrow      &               & \searrow \\
    f_2 := f[x_2] & \rightarrow   & f[x_1,x_2]    & \rightarrow   & f[x_0,...,x_2] 
    
\end{matrix}$
\subsection{Lagrange}
Die Lagrange-Darstellung besteht aus den St\"utzwerten als Koeffizienten und den Lagrange-Polynomen.
$p(x) = f_0 L_0(x) + f_1 L_1(x) + ... + f_n L_n(x)$
\subsection{Lagrange-Polynome}
$L_i(x) = \displaystyle\sum_{j=0,i\neq j}^n \frac{x - x_j}{x_0 - x_j}$


\rule{0.3\linewidth}{0.25pt}
\scriptsize

Copyright \copyright\ 2010 Christian Kniep \& Co

% Should change this to be date of file, not current date.
\verb!$Revision: 0.9 $, $Date: 2010/07/18$.!



\end{multicols}
\end{document}
